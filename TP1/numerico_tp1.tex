\documentclass[titlepage,a4paper]{article}

\usepackage{a4wide}
\usepackage[colorlinks=true,linkcolor=black,urlcolor=blue,bookmarksopen=true]{hyperref}
\usepackage{bookmark}
\usepackage{fancyhdr}
\usepackage[spanish]{babel}
\usepackage[utf8]{inputenc}
\usepackage[T1]{fontenc}
\usepackage{graphicx}
\usepackage{float}
\usepackage{multirow}

\pagestyle{fancy}
\fancyhf{}
\fancyhead[L]{TP1 - Grupo 5}
\fancyhead[R]{Análisis Numérico I - FIUBA}
\renewcommand{\headrulewidth}{0.4pt}
\fancyfoot[C]{\thepage}
\renewcommand{\footrulewidth}{0.4pt}


\begin{document}


\begin{titlepage}
	\hfill\includegraphics[width=6cm]{logofiuba.jpg}
    	\centering
    	\vfill
	\huge \textbf{Análisis Numérico I\\}
	\huge \textbf{[75.12/95.04]\\}
	\huge \textbf{Curso 3\\}
    	\huge \textbf{Trabajo Práctico 1}
    	\vskip2cm
	\large
	Grupo 5 \\
    	Primer cuatrimestre de 2019 
	\vfill

	\begin{tabular}{ |l|l|l| }
		\hline
		\multicolumn{3} { |c| } {\textbf{Integrantes del grupo}} \\ \hline
		Santamaría Tomás & padron & mail\\ \hline
	 	Hemmingsen Lucas & padron & mail\\ \hline
	 	Huenul Matías & 102135 & matias.huenul.07@gmail.com\\ \hline
	\end{tabular}
	\vfill
    	\vfill
\end{titlepage}


\section{Introducción}\label{sec:introd}


\section{Parte 1}\label{sec:parte1}
El segundo polinomio de Taylor de la función $ f(x) = e^xcos(x) $ alrededor de $ x_0 = \frac{\pi}{6} $ es:
	\begin{equation}
	P_2(x) = \frac{1}{2}\sqrt{3}e^{\frac{\pi}{6}} + \frac{1}{2}(\sqrt{3} - 1)e^{\frac{\pi}{6}}(x - \frac{\pi}{6}) - \frac{1}{2}e^{\frac{\pi}{6}}(x - \frac{\pi}{6})^2
	\end{equation}
Utilizando este polinomio para aproximar $ f(x) $ en $ x = 0.5 $ se obtiene:
	\begin{equation}
	f(0,5) \approx P_2(0,5) = 1,4468
	\end{equation}


\section{Parte 2}\label{sec:parte2}


\section{Parte 3}\label{sec:parte3}


\section{Referencias}\label{sec:parte4}


\end{document}