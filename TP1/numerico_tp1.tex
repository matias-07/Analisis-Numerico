\documentclass[titlepage,a4paper]{article}

\usepackage{a4wide}
\usepackage[colorlinks=true,linkcolor=black,urlcolor=blue,bookmarksopen=true]{hyperref}
\usepackage{bookmark}
\usepackage{fancyhdr}
\usepackage[spanish]{babel}
\usepackage[utf8]{inputenc}
\usepackage[T1]{fontenc}
\usepackage{graphicx}
\usepackage{float}
\usepackage{amsmath}
\usepackage{multirow}
\usepackage{enumitem}
\usepackage{listings}
\usepackage{color}
\usepackage{upgreek}
 
\definecolor{codegreen}{rgb}{0,0.6,0}
\definecolor{codegray}{rgb}{0.5,0.5,0.5}
\definecolor{codepurple}{rgb}{0.58,0,0.82}
\definecolor{backcolour}{rgb}{0.95,0.95,0.92}
 
\lstdefinestyle{mystyle}{
    backgroundcolor=\color{backcolour},   
    commentstyle=\color{codegreen},
    keywordstyle=\color{magenta},
    numberstyle=\tiny\color{codegray},
    stringstyle=\color{codepurple},
    basicstyle=\footnotesize,
    breakatwhitespace=false,         
    breaklines=true,                 
    captionpos=b,                    
    keepspaces=true,                 
    numbers=left,                    
    numbersep=5pt,                  
    showspaces=false,                
    showstringspaces=false,
    showtabs=false,                  
    tabsize=2
}
 
\lstset{style=mystyle}
\lstset{language=Octave}

\pagestyle{fancy}
\fancyhf{}
\fancyhead[L]{TP1 - Grupo 5}
\fancyhead[R]{Análisis Numérico I - FIUBA}
\renewcommand{\headrulewidth}{0.4pt}
\fancyfoot[C]{\thepage}
\renewcommand{\footrulewidth}{0.4pt}


\begin{document}


\begin{titlepage}
	\hfill\includegraphics[width=6cm]{logofiuba.jpg}
    	\centering
    	\vfill
	\huge \textbf{Análisis Numérico I\\}
	\huge \textbf{[75.12/95.04]\\}
	\huge \textbf{Curso 3\\}
    	\huge \textbf{Trabajo Práctico 1}
    	\vskip2cm
	\large
	Grupo 5 \\
    	Primer cuatrimestre de 2019 
	\vfill

	\begin{tabular}{ |l|l|l| }
		\hline
		\multicolumn{3} { |c| } {\textbf{Integrantes del grupo}} \\ \hline
		Santa María Tomás & 92797 & tomasisantamaria@gmail.com\\ \hline
	 	Hemmingsen Lucas & 76187 & lhemmingsen@fi.uba.ar\\ \hline
	 	Huenul Matías & 102135 & matias.huenul.07@gmail.com\\ \hline
	\end{tabular}
	\vfill
    	\vfill
\end{titlepage}


\section{Introducción}\label{sec:introd}
El objetivo del presente trabajo práctico es obtener aproximaciones de dos funciones y utilizarlas para calcular su valor en distintos puntos, estimando los errores cometidos en ambos casos y analizando las causas de los mismos en base a los conceptos teóricos vistos en el curso.


\section{Conceptos teóricos}\label{sec:conceptos}
\begin{description}
\item[Teorema de Taylor] Este teorema permite obtener aproximaciones polinomiales de funciones diferenciables en un cierto entorno, así también como una cota para el error de aproximación. Sea $ k \geq 1 $ un número entero y $ f(x) $ una función $k$ veces diferenciable en $ x_0 $, su polinomio de Taylor de orden $ k $ en torno a $  x_0 $ se define como:
	\begin{equation}
	P_k(x) = \sum_{i=0}^k \frac{f^{(i)}(x_0)}{i!}(x - x_0)^i\label{eq:1}
	\end{equation}
y el error que se comete aproximando $ f(x) $ a través de $ P_k(x) $ es:
	\begin{equation}
	R_k(x) = \frac{f^{(k+1)}(c)}{(k+1)!}(x - x_0)^{k+1}
	\end{equation}
donde $c \in [x_0, x]$ o $c \in [x, x_0]$. 
\end{description}


\section{Parte 1}\label{sec:parte1}
El segundo polinomio de Taylor de la función $ f(x) = e^xcos(x) $ alrededor de $ x_0 = \frac{\pi}{6} $ es:
	\begin{equation}
	P_2(x) = \frac{1}{2}\sqrt{3}e^{\frac{\pi}{6}} + \frac{1}{2}(\sqrt{3} - 1)e^{\frac{\pi}{6}}(x - \frac{\pi}{6}) - \frac{1}{2}e^{\frac{\pi}{6}}(x - \frac{\pi}{6})^2
	\end{equation}
Utilizando este polinomio para aproximar $ f(x) $ en $ x = 0.5 $ se obtiene:
	\begin{equation}
	f(0,5) \approx P_2(0,5) = 1,4469
	\end{equation}
El error estará dado por:
	\begin{equation}
	R_2(x) = \frac{-2e^{x}(sen(c) + cos(c))}{6}(x - \frac{\pi}{6})^3
	\end{equation}
Luego el error cometido al realizar la aproximación es:
	\begin{equation*}
	R_2(0,5) = 7,2226*10^{-6}(sen(c) + cos(c))
	\end{equation*}
	\begin{equation*}
	| R_2(0,5) | \leq 7,2226*10^{-6}(|sen(c)| + |cos(c)|) \leq 7,2226*10^{-6}*2
	\end{equation*}
	\begin{equation}
	| R_2(0,5) | \leq 1.4445*10^{-5}
	\end{equation}
La cota superior del error de aproximación de $ f(x) $ al usar $ P_2(x) $ en el intervalo $ [0, 1] $ es:
	\begin{equation*}
	| R_2 | \leq | \frac{-2e^{1}(sen(c) + cos(c))}{6}(1 - \frac{\pi}{6})^3 |
	\end{equation*}
	\begin{equation}
	| R_2 | \leq 0,19594
	\end{equation}
Al evaluar en Octave se obtiene $ f(0,5) = 1,4469 $. Este resultado coincide con el calculado mediante la aproximación, lo cual es esperable ya que de acuerdo a $(6)$, $P_2(x)$ logra aproximar a $f(x)$ hasta cuatro decimales sin error y por lo tanto, al redondear simétricamente como lo hace Octave, se llega al mismo valor.

A continuación se encuentra el código en \emph{Octave} de la implementación particular del segundo polinomio de Taylor para la función representada en el problema:
			\lstinputlisting[language=Octave]{taylor_ej1.m}

Este código imprime los siguientes resultados cuando se corre desde Octave:

\begin{lstlisting}
	>> taylor_ej1 (0.5)
	x0 =  0.52360
	f =  1.4619
	f1 =  0.61788
	f2 = -1.6881
	f3 = -4.5815
	p0 =  1.4619
	p1 = -0.014581
	p2 = -0.00047005
	error =  0.000010035
	ans =  1.4469
\end{lstlisting}

\section{Parte 2}\label{sec:parte2}
	\begin{enumerate}[label=(\alph*)]
		\item 
			El primer método que utilizaremos para aproximar la función error\cite{error_function} será desarrollar la serie de Taylor 
			de $e^{-x^2}$ alrededor de $x=0$ y luego aplicaremos la integral definida a cada término de la serie y 
			lo multiplicaremos por $2/\sqrt{\pi}$.\\
			Sabiendo que el desarrollo de Taylor de $e^x$ alrededor del cero es:
				\begin{equation}
					e^x = 1 + x + \frac{x^2}{2!} + \frac{x^3}{3!} + ... = \sum_{k=0}^{\infty}\frac{x^k}{k!}\label{eq:3}
				\end{equation}
			Sustituyendo para $e^{-x^2}$:
				\begin{equation}
					e^{-x^2} = 1 - x^2 + \frac{x^4}{2!} - \frac{x^6}{3!} + ... = \sum_{k=0}^{\infty}\frac{(-1)^k x^{2k}}{k!}
				\end{equation}
			Integrando cada término:
				\begin{equation}
					\int_{0}^{x}e^{-t^2}dt= x - \frac{x^3}{3} + \frac{x^5}{5\cdot2!} - \frac{x^7}{7\cdot3!} + ... = \sum_{k=0}^{\infty}\frac{(-1)^k x^{2k+1}}{(2k+1)\cdot k!}
				\end{equation}
			Y finalmente, multiplicando por $2/\sqrt{\pi}$ obtenemos:
				\begin{equation}
					erf(x) = \frac{2}{\sqrt{\pi}}\int_{0}^{x}e^{-t^2}dt = \frac{2}{\sqrt{\pi}}\sum_{k=0}^{\infty}\frac{(-1)^k x^{2k+1}}{(2k+1)\cdot k!}
				\end{equation}

			Como calcularemos un número limitado de términos de esta serie, tendremos un error de truncamiento, que estará dado por:


			Como segundo método, aproximaremos la integral de la función error con el método del trapecio\cite{burden1}. 
			En particular, utilizaremos el método del trapecio compuesto\cite{burden2}, con el que podremos ir variando 
			la precisión del resultado según la cantidad de subintervalos utilizados para calcular la integral.

			La regla del trapecio compuesta para $n$ subintervalos para una función $f(x)$ en un intervalo $[a,b]$ está dada por:
				\begin{equation}
					\int_{a}^{b}f(x)dx = \frac{h}{2}\left[f(a) + 2\sum_{j=1}^{n-1}f(x_{j})+f(b)\right]-\frac{b-a}{12}h^{2}f''(\mu)\label{eq:12}
				\end{equation}
			donde $h=(b-a)/n$ es el tamaño de cada subintervalo y $x_j = a + jh$ para cada $j=1,2,..,n$. El último 
			término corresponde al error del método donde $\mu \in (a,b)$.

			Para nuestra función error, siendo $f(x) = e^{-x^2}$ tenemos:
				\begin{equation}
					\int_{0}^{b}e^{-x^2}dx = \frac{h}{2}\left[1 + 2\sum_{j=1}^{n-1}e^{-{x_j}^2}+e^{-b^2}\right]-\frac{b}{12}h^{2}f''(\mu)
				\end{equation}
			Finalmente, multiplicando por $2/\sqrt{\pi}$ obtenemos nuestra segunda aproximación de la función error:
				\begin{equation}
					erf(x) = \frac{2}{\sqrt{\pi}}\int_{0}^{x}e^{-t^2}dt \simeq \frac{h}{\sqrt{\pi}}\left[1 + 2\sum_{j=1}^{n-1}e^{-{x_j}^2}+e^{-x^2}\right]
				\end{equation}
			donde $h = x/n$ y $x_j = jh$.
		
		\item
			A continuación se encuentra el código en \emph{Octave} de la aproximación de la función error mediante polinomios de Taylor:
			\lstinputlisting[language=Octave]{erf_taylor.m}
			
			La función para la aproximación utilizando el método de los trapecios implementada en \emph{Octave} es:
			\lstinputlisting[language=Octave]{erf_trapecios.m}

			Se pueden encontrar ambos archivos, \textit{erf\_taylor.m} y \textit{erf\_trapecios.m}, en la presente entrega.

			El módulo principal, \textit{ejercicio2.m}, realiza llamadas a ambos métodos variando el número de iteraciones para 
			cada método entre uno y diez, y calcula el error absoluto con respecto a la función \emph{erf} de \emph{Octave}.
			Como la aproximación por el método de los trapecios converge muy lentamente, se implementó una función que calcule el número 
			de iteraciones necesarias para lograr un número de cifras significativas, que es pasada por parámetros.
			Todos estos resultados se escriben en un archivo, \emph{resultados.txt}.
			\lstinputlisting[language=Octave]{ejercicio2.m}
		
		\item
			A continuación se muestran los resultados del archivo \emph{resultados.txt}:
			\lstinputlisting{resultados.txt}

	\end{enumerate}


\section{Parte 3}\label{sec:parte3}
	\begin{enumerate}[label=(\alph*)]
		
		\item
			Tanto en la parte 1 como en la parte 2 del presente Trabajo Práctico nos encontramos con distintas fuentes de error; a continuación 
			se describirán sus instancias de ocurrencia según la fuente de la cual estemos hablando.

			\begin{itemize}
				\item Errores Inherentes (EI): Son errores que existen en los datos de entrada del problema. En nuestro caso, al encontrarnos 
				con números irracionales en los parámetros de entrada ($\pi$ y $ e $), estos deben ser aproximados a un número finito de dígitos, 
				ya que no pueden representarse exactamente con la cantidad de dígitos significativos con los que nos encontramos trabajando. 
				En nuestro caso particular, Octave por defecto trabaja con 5 dígitos significativos.
				\item Errores por truncamiento (ED): Una fuente de error que se origina en la utilización de un algoritmo determinado. En el 
				caso del presente TP, los algoritmos utilizados en la parte 2 son una fuente de este tipo de error, ya que se utilizan una 
				cantidad finita de iteraciones para aproximar el resultado. Las iteraciones omitidas introducen un error de truncamiento; 
				en el caso de algoritmos con convergencia más lenta el error introducido será mayor (si la cantidad de iteraciones se mantiene).
				\item Errores por redondeo (ER): Error que se introduce al hacer cuentas con representaciones numéricas finitas. Cada operación 
				aritmética que realicemos en Octave puede incurrir en un error de redondeo debido a la metodología de redondeo que utiliza y la 
				cantidad de dígitos significativos empleados. El punto flotante es inherentemente una representación aproximada de un valor 
				decimal y por ello acarreará nativamente un error de redondeo. Además, cuantas más operaciones realicemos, mayor será el 
				arrastre del error; es por ello que los algoritmos empleados en la parte 2 corren un riesgo mayor con respecto a esta fuente de 
				error que el Polinomio de Taylor en la parte 1 del presente TP.
			\end{itemize}

		\item
			\begin{itemize}
				\item Problema Matemático (PM): En la parte 1, el problema matemático es la definición del Polinomio de Taylor presentada en la ecuación \emph{(1)}. En cuanto a la segunda parte, la instancia es la función error presentada en el enunciado.
				\item Problema Numérico (PN): Las aproximaciones presentadas en \emph{(3)} y \emph{(12)} representan las instancias de problema numérico del trabajo práctico.
				\item Método Numérico (MN): Los pasos necesarios para resolver el problema matemático e implementarlo en una computadora.
				\item Algoritmo: Los algoritmos presentados en \textit{erf\_taylor.m} y \textit{erf\_trapecios.m}, además de la resolución particular del polinomio existente en \textit{taylor\_ej1.m}, ya expuestos en el presente Trabajo Práctico.
			\end{itemize}

	\end{enumerate}

\begin{thebibliography}{9} 
	\bibitem{error_function}
		Error Function - Wikipedia, \url{https://en.wikipedia.org/wiki/Error_function}
	
	\bibitem{burden1}
		Richard L. Burden, J. Douglas Faires, \emph{Numerical Analysis (9th edition)} \\
		Brooks/Cole, Cengage Learning, 2010, Sección 4.3, página 194

	\bibitem{burden2}
	Richard L. Burden, J. Douglas Faires, \emph{Numerical Analysis (9th edition)} \\
	Brooks/Cole, Cengage Learning, 2010, Sección 4.4, página 206
	 
\end{thebibliography}



\end{document}