\documentclass[titlepage,a4paper]{article}

\usepackage{a4wide}
\usepackage[colorlinks=true,linkcolor=black,urlcolor=blue,bookmarksopen=true]{hyperref}
\usepackage{bookmark}
\usepackage{fancyhdr}
\usepackage[spanish]{babel}
\usepackage[utf8]{inputenc}
\usepackage[T1]{fontenc}
\usepackage{graphicx}
\usepackage{float}
\usepackage{amsmath}
\usepackage{multirow}

\pagestyle{fancy}
\fancyhf{}
\fancyhead[L]{TP1 - Grupo 5}
\fancyhead[R]{Análisis Numérico I - FIUBA}
\renewcommand{\headrulewidth}{0.4pt}
\fancyfoot[C]{\thepage}
\renewcommand{\footrulewidth}{0.4pt}


\begin{document}


\begin{titlepage}
	\hfill\includegraphics[width=6cm]{logofiuba.jpg}
    	\centering
    	\vfill
	\huge \textbf{Análisis Numérico I\\}
	\huge \textbf{[75.12/95.04]\\}
	\huge \textbf{Curso 3\\}
    	\huge \textbf{Trabajo Práctico 1}
    	\vskip2cm
	\large
	Grupo 5 \\
    	Primer cuatrimestre de 2019 
	\vfill

	\begin{tabular}{ |l|l|l| }
		\hline
		\multicolumn{3} { |c| } {\textbf{Integrantes del grupo}} \\ \hline
		Santa María Tomás & padron & mail\\ \hline
	 	Hemmingsen Lucas & padron & mail\\ \hline
	 	Huenul Matías & 102135 & matias.huenul.07@gmail.com\\ \hline
	\end{tabular}
	\vfill
    	\vfill
\end{titlepage}


\section{Introducción}\label{sec:introd}
El objetivo del presente trabajo práctico es obtener aproximaciones de dos funciones y utilizarlas para calcular su valor en distintos puntos, estimando los errores cometidos en ambos casos y analizando las causas de los mismos en base a los conceptos teóricos vistos en el curso.


\section{Conceptos teóricos}\label{sec:conceptos}
\begin{description}
\item[Teorema de Taylor] Este teorema permite obtener aproximaciones polinomiales de funciones diferenciables en un cierto entorno, así también como una cota para el error de aproximación. Sea $ k \geq 1 $ un número entero y $ f(x) $ una función $k$ veces diferenciable en $ x_0 $, su polinomio de Taylor de orden $ k $ en torno a $  x_0 $ se define como:
	\begin{equation}
	P_k(x) = \sum_{i=0}^k \frac{f^{(i)}(x_0)}{i!}(x - x_0)^i
	\end{equation}
y el error que se comete aproximando $ f(x) $ a través de $ P_k(x) $ es:
	\begin{equation}
	R_k(x) = \frac{f^{(k+1)}(c)}{(k+1)!}(x - x_0)^{k+1}
	\end{equation}
donde $c \in [x_0, x]$ o $c \in [x, x_0]$. 
\end{description}


\section{Parte 1}\label{sec:parte1}
El segundo polinomio de Taylor de la función $ f(x) = e^xcos(x) $ alrededor de $ x_0 = \frac{\pi}{6} $ es:
	\begin{equation}
	P_2(x) = \frac{1}{2}\sqrt{3}e^{\frac{\pi}{6}} + \frac{1}{2}(\sqrt{3} - 1)e^{\frac{\pi}{6}}(x - \frac{\pi}{6}) - \frac{1}{2}e^{\frac{\pi}{6}}(x - \frac{\pi}{6})^2
	\end{equation}
Utilizando este polinomio para aproximar $ f(x) $ en $ x = 0.5 $ se obtiene:
	\begin{equation}
	f(0,5) \approx P_2(0,5) = 1,4469
	\end{equation}
El error estará dado por:
	\begin{equation}
	R_2(x) = \frac{-2e^{x}(sen(c) + cos(c))}{6}(x - \frac{\pi}{6})^3
	\end{equation}
Luego el error cometido al realizar la aproximación es:
	\begin{equation*}
	R_2(0,5) = 7,2226*10^{-6}(sen(c) + cos(c))
	\end{equation*}
	\begin{equation*}
	| R_2(0,5) | \leq 7,2226*10^{-6}(|sen(c)| + |cos(c)|) \leq 7,2226*10^{-6}*2
	\end{equation*}
	\begin{equation}
	| R_2(0,5) | \leq 1.4445*10^{-5}
	\end{equation}
La cota superior del error de aproximación de $ f(x) $ al usar $ P_2(x) $ en el intervalo $ [0, 1] $ es:
	\begin{equation*}
	| R_2 | \leq | \frac{-2e^{1}(sen(c) + cos(c))}{6}(1 - \frac{\pi}{6})^3 |
	\end{equation*}
	\begin{equation}
	| R_2 | \leq 0,19594
	\end{equation}
Al evaluar en Octave se obtiene $ f(0,5) = 1,4469 $. Este resultado coincide con el calculado mediante la aproximación, lo cual es esperable ya que de acuerdo a $(6)$, $P_2(x)$ logra aproximar a $f(x)$ hasta cuatro decimales sin error y por lo tanto, al redondear simétricamente como lo hace Octave, se llega al mismo valor.

\section{Parte 2}\label{sec:parte2}


\section{Parte 3}\label{sec:parte3}


\section{Referencias}\label{sec:parte4}


\end{document}