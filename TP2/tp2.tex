\documentclass[titlepage,a4paper]{article}

\usepackage{a4wide}
\usepackage[colorlinks=true,linkcolor=black,urlcolor=blue,bookmarksopen=true]{hyperref}
\usepackage{bookmark}
\usepackage{fancyhdr}
\usepackage[spanish]{babel}
\usepackage[utf8]{inputenc}
\usepackage[T1]{fontenc}
\usepackage{graphicx}
\usepackage{float}
\usepackage{amsmath}
\usepackage{multirow}
\usepackage{enumitem}
\usepackage{listings}
\usepackage{color}
\usepackage{upgreek}
 
\definecolor{codegreen}{rgb}{0,0.6,0}
\definecolor{codegray}{rgb}{0.5,0.5,0.5}
\definecolor{codepurple}{rgb}{0.58,0,0.82}
\definecolor{backcolour}{rgb}{0.95,0.95,0.92}
 
\lstdefinestyle{mystyle}{
    backgroundcolor=\color{backcolour},   
    commentstyle=\color{codegreen},
    keywordstyle=\color{magenta},
    numberstyle=\tiny\color{codegray},
    stringstyle=\color{codepurple},
    basicstyle=\footnotesize,
    breakatwhitespace=false,         
    breaklines=true,                 
    captionpos=b,                    
    keepspaces=true,                 
    numbers=left,                    
    numbersep=5pt,                  
    showspaces=false,                
    showstringspaces=false,
    showtabs=false,                  
    tabsize=2
}
 
\lstset{style=mystyle}
\lstset{language=Octave}

\pagestyle{fancy}
\fancyhf{}
\fancyhead[L]{TP2 - Grupo 5}
\fancyhead[R]{Análisis Numérico I - FIUBA}
\renewcommand{\headrulewidth}{0.4pt}
\fancyfoot[C]{\thepage}
\renewcommand{\footrulewidth}{0.4pt}


\begin{document}


\begin{titlepage}
	\hfill\includegraphics[width=6cm]{logofiuba.jpg}
    	\centering
    	\vfill
	\huge \textbf{Análisis Numérico I\\}
	\huge \textbf{[75.12/95.04]\\}
	\huge \textbf{Curso 3\\}
	\huge \textbf{Trabajo Práctico 2}\\
	\huge \textbf{Problema de los Tres Cuerpos Restringido}
	\vskip2cm
	\large
	Grupo 5 \\
    	Primer cuatrimestre de 2019 
	\vfill

	\begin{tabular}{ |l|l|l| }
		\hline
		\multicolumn{3} { |c| } {\textbf{Integrantes del grupo}} \\ \hline
		Santa María Tomás & 92797 & tomasisantamaria@gmail.com\\ \hline
	 	Hemmingsen Lucas & 76187 & lhemmingsen@fi.uba.ar\\ \hline
	 	Huenul Matías & 102135 & matias.huenul.07@gmail.com\\ \hline
	\end{tabular}
	\vfill
    	\vfill
\end{titlepage}

\tableofcontents %Esta línea genera un indice a partir de las secciones y subsecciones creadas en el documento
	\newpage

	
\section{Introducción}\label{sec:introd}
	El objetivo del presente trabajo práctico es utilizar los distintos metodos numéricos de problema de valor inicial 
	para ecuaciones diferenciales ordinarias, para resolver el ``Problema de los Tres Cuerpos Restringido o de Euler'' 
	y realizar una comparación de los resultados con cada método.

	Los métodos que utilizaremos son el método de Euler, Runge-Kutta de orden 2 y 4, Nystr\"om y Newmark.

	


\section{Conceptos teóricos}\label{sec:conceptos}
	\subsection{Método de Euler}
	Sea $ y'(t) = f(t, y) $ una ecuación diferencial ordinaria de primer orden con condición inicial $ y(t_{0}) = y_{0} $. Se quiere obtener la solución en $ N + 1 $ puntos en un intervalo dado, uniformemente espaciado con paso $h$. 
	Se puede aproximar y(t) por su Polinomio de Taylor de primer orden:
	\begin{equation}
		y(t_{k} + h) \cong y(t_{k}) + hy'(t_{k})
	\end{equation}
	Pero como $y'(t_{k}) = f(t_{k}, y(t_{k}))$ y además $ t_{k+1} = t_{k} + h $ para todo $ 0 \leq k \leq N $, esta expresión se puede reescribir como:
	\begin{equation}
		y(t_{k+1}) \cong t(t_{k}) + hf(t_{k}, y(t_{k}))
	\end{equation}
	Este método puede usarse para resolver sistemas de ecuaciones diferenciales,  aplicando lo anterior a cada componente del sistema.	
	\subsection{Método de Runge-Kutta}
	\subsection{Método de Nystr\"om}
	\subsection{Método de Newmark}

	\subsection{Sistema de ecuaciones diferenciales de primer orden}\cite{burden_system_diff_eq}
		Sea un sistema de ecuaciones diferenciales de primer orden con la forma:

		\begin{equation}
			\label{eq1}
			\begin{cases}
				\frac{du_{1}}{dt}=f_{1}(t,u_{1},u_{2},...,u_{m})\\
				\frac{du_{2}}{dt}=f_{2}(t,u_{1},u_{2},...,u_{m})\\
				\vdots\\
				\frac{du_{m}}{dt}=f_{m}(t,u_{1},u_{2},...,u_{m})
			\end{cases}
		\end{equation}

		para $a \leq t \leq b$, con condiciones iniciales:
		\begin{equation}
			\label{eq2}
			u_{1}(a) = \alpha_{1}, u_{2}(a)=\alpha_{2}, \dots, u_{m}(a)=\alpha_{m}
		\end{equation}
		El objetivo es encontrar \emph{m} funciones $u_{1}(t),u_{2}(t),\dots,u_{m}(t)$ que satisfacen 
		cada una de las ecuaciones diferenciales junto con las condiciones iniciales.
		Los métodos para resolver sistemas de ecuaciones diferenciales de primer orden son simplemente 
		generalizaciones de los métodos para una sola ecuación diferencial presentados anteriormente.

	\subsection{Ecuaciones diferenciales de orden mayor a 1}\cite{burden_higher_order_diff_eq}
			Una ecuación diferencial de grado \emph{m}
			\begin{equation}
				y^{(m)}(t)=f(t,y,y',\dots,y^{(m-1)})
			\end{equation}
			con $a \leq t \leq b$ y condiciones iniciales $y(a)=\alpha_{1},y'(a)=\alpha_{2},\dots,y^{(m-1)(a)=\alpha_{m}}$ 
			puede convertirse a un sistema de ecuaciones de la forma de \eqref{eq1} y \eqref{eq2}.

			Sea $u_{1}(t)=y(t),u_{2}(t)=y'(t),\dots,u_{m}(t)=y^{(m-1)}(t)$. Esto produce el sistema de ecuaciones diferenciales 
			de primer orden

			\begin{equation}
				\frac{du_{1}}{dt}=\frac{dy}{dt}=u_{2},\frac{du_{2}}{dt}=\frac{dy'}{dt}=u_{3},\dots,\frac{du_{m-1}}{dt}=\frac{dy^{(m-2)}}{dt}=u_{m}
			\end{equation}

			y

			\begin{equation}
				\frac{du_{m}}{dt}=\frac{dy^{(m-1)}}{dt}=y^{(m)}=f(t,y,y',\dots,y^{(m-1)})=f(t,u_{1},u_{2},\dots,u_{m})
			\end{equation}

			con condiciones iniciales

			\begin{equation}
				u_{1}(a)=y(a)=\alpha_{1}, u_{2}(a)=y'(a)=\alpha_{2},\dots,u_{m}(a)=y^{(m-1)}(a)=\alpha_{m}
			\end{equation}
\section{Desarrollo}\label{sec:desarrollo}
	\subsection{Parte A}\label{sec:parteA}

		Tenemos el siguiente sistema de ecuaciones diferenciales de segundo orden que representan el movimiento de un satélite
		viajando entre la tierra y la luna e influenciado gravitatoriamente solo por estos dos cuerpos:
		\begin{equation}
			\begin{cases}
				x''_{1} = 2x'_{2} + x_{1} - \eta\frac{x_{1} + \mu}{d_{1}^{3}} - \mu\frac{x_{1}-\eta}{d_{2}^{3}}\\
				x''_{2} = -2x'_{1} + x_{2} - \eta\frac{x_{2}}{d_{1}^{3}} - \mu\frac{x_{2}}{d_{2}^{3}}
			\end{cases}
		\end{equation}

		Siendo $d_{1}=\sqrt{(x_{1}+\mu)^{2} + x_{2}^{2}}$ y $d_{2}=\sqrt{(x_{1}-\eta)^{2} + x_{2}^{2}}$.

		Sea:

		\begin{equation}
			\begin{cases}
				v_{1}(t) = x'_{1}(t)\\
				v_{2}(t) = x'_{2}(t)
			\end{cases}
		\end{equation}

		Entonces podemos transformar el sistema de ecuaciones anterior en un sistema de cuatro ecuaciones 
		diferenciales de primer orden:

		\begin{equation}
			\begin{cases}
				x'_{1}(t) = v_{1}(t)\\
				v'_{1}(t) = 2v_{2}(t) + x_{1}(t) - \eta\frac{x_{1}(t) + \mu}{d_{1}(t)^{3}} - \mu\frac{x_{1}(t) - \eta}{d_{2}(t)^{3}}\\
				x'_{2}(t) = v_{2}(t)\\				
				v'_{2}(t) = -2v_{1}(t) + x_{2}(t) - \eta\frac{x_{2}(t)}{d_{1}(t)^{3}} - \mu\frac{x_{2}(t)}{d_{2}(t)^{3}}
			\end{cases}
		\end{equation}

		Con valores iniciales en $t=t_{0}$:

		\begin{equation}
			\begin{cases}
				x_{1}(t_{0}) = x_{1_{0}}\\
				v_{1}(t_{0}) = v_{1_{0}}\\
				x_{2}(t_{0}) = x_{2_{0}}\\				
				v_{2}(t_{0}) = v_{2_{0}}\\
			\end{cases}
		\end{equation}

	\subsection{Parte B}\label{sec:parteB}
		Ahora resolvamos el problema numéricamente con la función \emph{lsode} de \emph{Octave}.		
		Primero creamos la función \emph{yprima}, que representa a la función \emph{f(t,y)}.
		A continuación se detalla el código de la misma:

		\lstinputlisting[language=Octave]{yprima.m}

		Luego, desde la consola de \emph{Octave} ejecutamos la función \emph{lsode} con la función 
		\emph{yprima} como entrada, con una posición inicial $(x_{1}, x_{2}) = (1.2, 0)$ y velocidad inicial 
		$(v_{1}, v_{2}) = (0, -0.8)$ en el intervalo $[t_{0}, t_{1}) = [0, 2]$ con un $h=0.01$.
		Para ello, ejecutamos:

			\begin{lstlisting}
				[y]=lsode('yprima',[1.2 0 0 -0.8],0:0.01:2)			
			\end{lstlisting}
		con lo que en la última iteración (en t=2) obtenemos una posición final 
		$(x_{1}, x_{2}) = (-0.51306, 0.07881)$ y una velocidad final $(v_{1}, v_{2}) = (-1.18383, -0.48564)$.

		A continuación se muestra un gráfico de la trayectoria del satélite, que se mueve desde el extremo
		derecho del gráfico hacia la izquierda.
		
		\includegraphics[width=\textwidth]{parteb.png}

	\subsection{Parte C}\label{sec:parteC}
	Veremos ahora como se modifica la trayectoria aumentando el paso $h$. Para ello implementamos siguiente función.\\

	\lstinputlisting[language=Octave]{lsode_con_distintos_h.m}

	En el siguiente gráfico se puede observar lo obtenido para cuatro valores  de $h$ en particular, utilizando como condiciones iniciales las mismas que al comienzo de la parte b.\\
	\includegraphics[width=\textwidth]{partec.png}

	\subsection{Parte D}\label{sec:parteD}
	Resolveremos ahora el sistema en el intervalo $[0; 2]$ mediante el método de Euler. Las ecuaciones son:
	\begin{equation}
		\begin{cases}
			x_1 (t_{k+1}) = x_1 (t_{k}) + hv_1 (t_{k})\\
			v_1 (t_{k+1}) = x_2 (t_{k}) + h(2v_2 (t_{k}) + x_1 (t_{k}) - \eta\frac{x_1 (t_{k}) - \mu}{d_1 (t_{k})^{3}})\\
			x_2 (t_{k+1}) = x_2 (t_{k}) + hv_2 (t_{k})\\
			v_2 (t_{k+1}) = v_2 (t_{k}) + h(-2v_1 (t_{k}) + x_2 (t_{k}) - \eta\frac{x_2 (t_{k})}{d_1 (t_{k})^{3}}) - \mu\frac{x_2 (t_{k})}{d_2 (t_{k})^{3}})\\
		\end{cases}
	\end{equation}

	El código de la función implementada en \emph{Octave} para la resolución con este método se muestra a continuación.
	\lstinputlisting[language=Octave]{euler.m}
	Primero utilizamos como condición inicial el vector $ y_0 = (1,2; 0; 0; -0,8) $ y como tolerancia el valor $ h = 0.01 $.\\
	Los resultados para cada $t$ del intervalo se encuentran en el archivo \emph{resultados-parte-d.txt}.\\
	En particular, el resultado obtenido para $t=2$ es:
	\begin{equation}
		y(2) = (-0.54958; -1.49872; 0.21286; -0.03654)
	\end{equation}
	La trayectoria obtenida en este caso es la siguiente.\\
	\includegraphics[width=\textwidth]{parted.png}

	\subsection{Parte E}\label{sec:parteE}

	\subsection{Parte F}\label{sec:parteF}

	\subsection{Parte G}\label{sec:parteG}

\section{Conclusiones}\label{sec:conc}


\begin{thebibliography}{9} 
	\bibitem{burden_system_diff_eq} 
		Richard L. Burden, J. Douglas Faires\\
		\textit{Numerical Analysis (9th edition)}. \\
		Brooks/Cole - Cengage Learning\\
		Sección 5.9 Higher Order Equations and Systems of Differential Equations\\
		Páginas 328-334

	\bibitem{burden_higher_order_diff_eq} 
		Richard L. Burden, J. Douglas Faires\\
		\textit{Numerical Analysis (9th edition)}.\\ 
		Brooks/Cole - Cengage Learning\\
		Sección 5.9 Higher Order Equations and Systems of Differential Equations\\
		Páginas 334-336
	 
\end{thebibliography}



\end{document}